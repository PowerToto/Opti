\documentclass{article}

% Language setting
% Replace `english' with e.g. `spanish' to change the document language
\usepackage[english]{babel}

% Set page size and margins
% Replace `letterpaper' with `a4paper' for UK/EU standard size
\usepackage[letterpaper,top=2cm,bottom=2cm,left=3cm,right=3cm,marginparwidth=1.75cm]{geometry}

% Useful packages
\usepackage{amsmath}
\usepackage{graphicx}
\usepackage[colorlinks=true, allcolors=blue]{hyperref}

\title{Devoir d’Optimisation}
\author{Aéna ARIA \and Thomas WIECZOREK}
\date{}
\begin{document}
\maketitle
\textbf{\begin{center}
    
\end{center}
}

\textbf{
\section{Enumération
}
}
On énumère toutes les solutions possibles et on prend celle qui satisfait le plus de monde 

Pour obtenir le nombre de recettes avec N ingrédients on prend toutes les recettes a un ingrédient  (combinaison 1 parmi N) + les recettes a deux ingrédients (2 parmi N) … jusqu’à N parmi N

donc on a : 
\[ \sum_{i=1}^{N} \frac{N!}{i!(N-i)!}  recettes\] 


Pour A,B et C on peut évaluer chaque recette une par une et prendre celle avec le meilleur score

Sur E on a 10000 ingrédients

on aurait \[ \sum_{i=1}^{10000} \frac{N!}{i!(N-i)!} recettes\] 

      pour évaluer on va devoir passer chaque recette sur chaque client
        chaque itération coûte une évaluation (itération sur les deux listes du client L1 et L2)
     \[\sum_{i=1}^{N=10000} \frac{N!}{i!(N-i)!}*\sum_{i=1}^{4986} L1 + L2\]
on peut faire $10^9$ opérations par seconde donc \[\frac{\sum_{i=1}^{N=10000} \frac{N!}{i!(N-i)!} * \sum_{i=1}^{4986} L1 + L2}{10^9} secondes\]
\textbf{
\section{Algorithme génétique
}}
-On modèlise les recettes par une liste de bits (1 si on prend l'ingrédient 0 sinon ), (au début on utilisait des listes de strings mais elles étaient trop lentes)

-On prend une population de 100

-Pour le croisement on coupe en un point aléatoire

-Pour la mutation on effectue un bit flip (on prend un ingrédient aléatoire on le rajoute si il était absent, on le retire si il était présent)

-On s'arrête quand on a 100 générations sans amélioration

\textbf{
\section{Recuit Simulé
}}
\end{document}

