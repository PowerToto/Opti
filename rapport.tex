\documentclass{article}

% Language setting
% Replace `english' with e.g. `spanish' to change the document language
\usepackage[english]{babel}

% Set page size and margins
% Replace `letterpaper' with `a4paper' for UK/EU standard size
\usepackage[letterpaper,top=2cm,bottom=2cm,left=3cm,right=3cm,marginparwidth=1.75cm]{geometry}

% Useful packages
\usepackage{amsmath}
\usepackage{graphicx}
\usepackage[colorlinks=true, allcolors=blue]{hyperref}

\title{Devoir d’Optimisation}
\author{Aéna ARIA \and Thomas WIECZOREK}
\date{}
\begin{document}
\maketitle
\textbf{\begin{center}
    
\end{center}
}

\textbf{
\section{Enumération
}
}
On énumère toutes les solutions possibles et on prend celle qui satisfait le plus de monde 

Pour obtenir le nombre de recettes avec N ingrédients on prend toutes les recettes a un ingrédient  (combinaison 1 parmi N) + les recettes a deux ingrédients (2 parmi N) … jusqu’à N parmi N

donc on a : 
\[ \sum_{i=1}^{N} \frac{N!}{i!(N-i)!}  recettes\] 


Pour A,B et C on peut évaluer chaque recette une par une et prendre celle avec le meilleur score

Sur E on a 10000 ingrédients

on aurait 10000*(10000+10000)
          (somme) (N!) (N-i)!*i!

      donc environ 10000*20000 = 8 000 000 de recettes

      pour évaluer on va devoir passer chaque recette sur chaque client
        chaque itération coûte une évaluation (itération sur les deux listes du client L1 et L2)
      on \[8 000 000 *\sum_{i=1}^{4986} L1 + L2\]
on peut faire $10^9$ opérations par seconde donc \[\frac{8 000 000 * \sum_{i=1}^{4986} L1 + L2}{10^9}\]secondes
\end{document}