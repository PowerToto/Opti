\documentclass{article}

% Language setting
% Replace `english' with e.g. `spanish' to change the document language
\usepackage[english]{babel}

% Set page size and margins
% Replace `letterpaper' with `a4paper' for UK/EU standard size
\usepackage[letterpaper,top=2cm,bottom=2cm,left=3cm,right=3cm,marginparwidth=1.75cm]{geometry}

% Useful packages
\usepackage{amsmath}
\usepackage{graphicx}
\usepackage[colorlinks=true, allcolors=blue]{hyperref}

\title{Devoir d’Optimisation}
\author{Aéna ARIA \and Thomas WIECZOREK}
\date{}
\begin{document}
\maketitle
\textbf{\begin{center}
    
\end{center}
}
Pour modèliser les recettes on utilisait une liste de strings, notamment pour les énumérations, mais cette version était trop lente donc nous avon basculé sur des listes de bits pour les métaheuristiques
\textbf{
\section{Enumération
}
}
On énumère toutes les solutions possibles et on prend celle qui satisfait le plus de monde 

Pour obtenir le nombre de recettes avec N ingrédients on prend toutes les recettes a un ingrédient  (combinaison 1 parmi N) + les recettes a deux ingrédients (2 parmi N) … jusqu’à N parmi N

donc on a : 
\[ \sum_{i=1}^{N} \frac{N!}{i!(N-i)!}  recettes\] 


Pour A,B et C on peut évaluer chaque recette une par une et prendre celle avec le meilleur score

Sur E on a 10000 ingrédients

on aurait \[ \sum_{i=1}^{10000} \frac{N!}{i!(N-i)!} recettes\] 

      pour évaluer on va devoir passer chaque recette sur chaque client
        chaque itération coûte une évaluation (itération sur les deux listes du client L1 et L2)
     \[\sum_{i=1}^{N=10000} \frac{N!}{i!(N-i)!}*\sum_{i=1}^{4986} L1 + L2\]
on peut faire $10^9$ opérations par seconde donc \[\frac{\sum_{i=1}^{N=10000} \frac{N!}{i!(N-i)!} * \sum_{i=1}^{4986} L1 + L2}{10^9} secondes\]
\textbf{
\section{Algorithme génétique
}}
On prend une population de 100, pour former les enfants on coupe les parents en un point aléatoire et on mélange la première partie de la pizza1 avec la deuxième partie de la pizza2 et inversement. On applique aussi une mutation (probabilité de 0.03) où on effectue un bitflip :on prend un ingrédient aléatoire on le rajoute si il était absent, on le retire si il était présent.

On s'arrête quand on atteint 100 générations sans amélioration

Score obtenu pour D: environ 1350\newline
Score obtenu pour E: environ 350
                
\textbf{
\section{Recuit Simulé
}}
On part d'une recette génerée aléatoirement, que l'on va altérer avec un bitflip entre chaque solution (toujours si on obtient une descente de gradient ou avec une proba de \exp(-\Delta score/T)).

La temperature choisie est 100°, on décroit de 10 à chaque itération\%  et on s'arrête lorsque la temperature atteint 0.1°

Score obtenu pour D environ 1250\newline
Score obtenu pour E: environ 350

                
\textbf{
\section{Recherche Tabou
}}

On part d'une configuration initiale qui sera une pizza choisie aléatoirement et d'une liste tabou d'abord vide.

On effectue un certain nombre de mouvements (bitflips) qui ne sont pas dans la liste tabou


On s'arrête au bout d'un certain nombre d'itérations sans amélioration
\end{document}

